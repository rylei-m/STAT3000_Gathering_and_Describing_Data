\documentclass{article}
\usepackage{graphicx} % Required for inserting images
\usepackage{amsmath}

\title{Gathering and Describing Data Homework }
\author{Rylei Mindrum}
\date{October 2024}

\begin{document}

\maketitle

\section{Introduction}

\section*{6.1.1 Rolls of a Six-Sided Die}
\begin{itemize}
    \item (a) \textbf{Population:} The set of all possible outcomes from rolling a six-sided die.
    \newline \textbf{Representative sample:} The sample is representative if the die is fair and shaken properly, ensuring each outcome is equally likely.
    \item (b) \textbf{Other factors:} One factor to consider is whether the die is biased or whether there is any mechanical influence affecting the outcomes. The rolling method could also be a source of concern.
\end{itemize}

\section*{6.1.3 Eye Colors of Programming Students}
\begin{itemize}
    \item (a) \textbf{Population:} The eye colors of all the students who are registered for the programming course.
    \newline \textbf{Representative sample:} The sample may not be representative of the general population as it is limited to students taking a programming course, which may not reflect broader demographics.
    \item (b) \textbf{Other factors:} It would be important to consider if certain demographics are over represented or underrepresented in the class, which could influence the interpretation of the results.
\end{itemize}

\section*{6.1.5 Fruit Spoilage}
\begin{itemize}
    \item (a) \textbf{Population:} Boxes of peaches received by the supermarket during the summer.
    \newline \textbf{Representative sample:} The random selection of one box per shipment could provide a representative sample, assuming that spoilage is uniform across boxes.
    \item (b) \textbf{Other factors:} If the spoilage rate varies across shipments, or if certain conditions like weather or handling, impact specific shipments more than others.
\end{itemize}

\section*{6.1.7 Paving Slab Weights}
\begin{itemize}
    \item (a) \textbf{Population:} paving slabs produced by the company.
    \newline \textbf{Representative sample:} If the slabs are randomly selected, the sample could be representative of the shipment; however, further investigation is needed to confirm if the shipment is uniform.
    \item (b) \textbf{Other factors:} The weights may vary due to manufacturing inconsistencies, and it may be useful to investigate the consistency of slab weights across different batches or shipments.
\end{itemize}

\section*{6.1.9 Plastic Panel Bending Capabilities}
\begin{itemize}
    \item (a) \textbf{Population:} all of the plastic panels produced by the machine.
    \newline \textbf{Representative sample:} The sample of 80 randomly selected panels could be representative of the entire population, assuming the panels are produced under consistent conditions.
    \item (b) \textbf{Other factors:} It would be important to verify whether the machine's production conditions are stable over time, variations could affect the deformity angles of the panels.
\end{itemize}

\section*{6.1.11 Market Research on New Product}
\begin{itemize}
    \item B.
\end{itemize}

\section*{6.2.7 Eye Colors}
\begin{itemize}
    \item \textbf{Graphical Presentations:} 
    \begin{itemize}
        \item A bar chart  - used to represent the frequencies of each eye color.
        \item A pie chart - used to show the proportions of eye colors in the sample.
    \end{itemize}
    \item \textbf{Outliers:} the data set represents categorical data, there are no numeric outliers, but any eye colors that occur very infrequently could be considered unusual like"other" from the data set.
    \item \textbf{Analysis:} The graphical presentation reveals certain eye colors are more common within a group, and indicates that the distribution of eye colors is skewed away from "other".
\end{itemize}

\section*{6.2.9 Fruit Spoilage}
\begin{itemize}
    \item \textbf{Graphical Presentations:} 
    \begin{itemize}
        \item histogram - the number of spoiled peaches per box over the 55 days could show the distribution of spoilage.
        \item  box plot - provides a view of the spread of the data and helps identify outliers 
    \end{itemize}
    \item \textbf{Outliers:} 25 and 14
    \item \textbf{Analysis:} the graphs help identify patterns, like whether spoilage is consistent over time or if certain days have significantly higher spoilage rates. outliers suggest problems with certain shipments.
\end{itemize}

\section*{6.2.11 Paving Slab Weights}
\begin{itemize}
    \item \textbf{Graphical Presentations:} 
    \begin{itemize}
        \item histogram of the paving slab weights shows the overall distribution of weights.
        \item  box plot helps identify the spread of the data and reveal potential outliers in terms of unusually light or heavy slabs.
    \end{itemize}
    \item \textbf{Outliers:} .538 (the largest observation)
    \item \textbf{Analysis:} The histogram and box plot reveal whether the normal distribution and skewness. the outliers suggests inconsistencies in production/measurement.
\end{itemize}

\section*{6.2.13 Plastic Panel Bending Capabilities}
\begin{itemize}
    \item \textbf{Graphical Presentations:} 
    \begin{itemize}
        \item Histogram  - helps visualize the distribution of bending capabilities.
        \item Box plot - highlights the central tendency, spread, and any potential outliers in the deformity angles.
    \end{itemize}
    \item \textbf{Outliers:} negatively skewed data
    \item \textbf{Analysis:} The graphical representations show that the bending capabilities are negatively skewed across the panel. 
\end{itemize}

\section*{6.3.4 Die Rolls}
\begin{itemize}
    \item \textbf{Graphical Presentations:} 
    \begin{itemize}
        \item bar chart shows the frequency of each outcome (1 through 6)
        \item Probability distribution plot  - useful for visualizing whether the die is fair
    \end{itemize}
    \item \textbf{Results:} mean: x-3.57, mediam: 3.5, t mean: 3.57, sd: s-1.77, uq: 5, lq: 2
\end{itemize}

\section*{6.3.6 Fruit Spoilage}
\begin{itemize}
    \item \textbf{Graphical Presentations:} 
    \begin{itemize}
        \item Histogram shows the number of spoiled peaches in the sample boxes across 55 days which shows the distribution of spoilage.
        \item Time series plot to observe spoilage trends over time.
    \end{itemize}
    \item \textbf{Results:} mean: x-3.29, median: 2, t mean: 2.76, sd: s-3.794, uq: 4, lq: 1

\end{itemize}

\section*{6.3.8 Paving Slab Weights}
\begin{itemize}
    \item \textbf{Graphical Presentations:} 
    \begin{itemize}
Hhistogram - shows the distribution of paving slab weights helps assess whether the weights follow a normal distribution.
        \item A box plot will help visualize the spread of the data and identify potential outliers.
    \end{itemize}
    \item     \textbf{Results:} mean: x-1.111, median: 1.110, t mean: 1.111, sd: s-.053, uq: 1.14, lq: 1.08
\end{itemize}

\section*{6.3.16 Skewness Interpretation}
\begin{itemize}
    \item \textbf{Question:} If a histogram is skewed with a long left tail, which of the following must be correct?
    \begin{itemize}
        \item B
        \end{itemize}
\end{itemize}
\end{document}
